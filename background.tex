\section{Background}
In order to achieve Model Predictive Control, it\textquotesingle s important to build a valid model of the dynamics of the used RC car. In order to do so, it\textquotesingle s important to understand the tire characteristics of this vehicle. This means there has to be a model for the tire characteristics as well. To build such a model, Data-Driven Model Design is used. 
To analyse the data of the test rig, a dynamical model is necessary. For this case, the Bicycle Model is chosen. 

In the Bicycle Model, the car is represented as a rigid, two-dimensional, two-wheel vehicle. The front wheels are represented as one wheel and so are the rear wheels. Nevertheless, this model comes with some important assumptions so there are some restrictions to test settings as well. The car should not move in the vertical (z-) position, nor rotate in the pitch and roll directions (around the x- and y- axles). Furthermore, the longitudinal velocity needs to be constant for calculations in the linear regime.
