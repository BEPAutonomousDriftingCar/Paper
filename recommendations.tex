\section{Recommendations}
As seen in previous sections decent results were already accomplished, however area of improvements are readily found and much more can be done in this area. First of all many (physical) improvements could be made to the testbed. Moreover, the signal analysis part of this research leaves possibilities for improvement. Last, further steps can be taken towards implementation of this research into autonomous or assisting driver systems can be taken.

\subsection{Testbed}
As mentioned before, the current testbed leaves room for improvement. As it\textquotesingle s been used for 2 full projects now, the car is showing his age and repeatedly being run into the wall or other impact related accidents have taken it's toll on the frame. As it is, the frame is now warped and and requires an offset on the front wheels to drive straight again at the lower power ranges. However, when driving in the high power ranges the vehicle dynamics become different and the offset will eventually make the car turn, or so we found out from our experiments. Although this offset can be corrected in the measurements the fact that it is need suggest there are other problems with the data being generated. For instance that the rear wheels are out of alignment. 

Moreover, there are quite some troubles with vibrations. Seeing as the regular dampeners are removed to generate a more modellike car vibrations effect the car heavily. The IMU unfortunately, is highly susceptible to this noise. Reducing vibrations should be high on the priority list when wanting to improve the results of this research.

Furthermore, the 3d printed wheels are hard to balance as the tolerance on the construction method is quite high. The result of this are even bigger vibrations than normal. This is further enlarged by placing the magnets which have even high tolerances. Seeing as a final design has ben made at the end of this research. Some effort to improve results might be aimed towards building proper balanced wheels using more precise techniques as a lathe, wheel balancing tools and maybe even custom magnets.
\subsection{Signal analysis}
Another area of improvement is the signal analysis. During this research an attempt was made into the spectrum of noise experienced on the IMU. However, it was difficult to determine the spectrum of noise from the actual signal as the fourier spectrum wasn't clear on this. There a filter was created with a cut off frequency of 5hz. We determined that Vibrations from the wheels would be generated from 5,5hz onwards and therefore settled on this frequency. Further signal analysis might improve on this rudimentary constructed filter by doing a proper work up of the signals generated. When combined with Noise (vibration) reduction. The signal might be cleaned up significantly to give a more clear fourier spectrum on the noise and the actual signal. Therefore allowing a reduced filter, which might generate a better picture through more data points.  
\subsection{Implementation}
Lastly we look to implementation. The eventual aim of this research is to be able to use this in nonlinear model predictive control and eventually implementation into cars. However, when looking towards this future it\textquotesingle s evident a couple of problems may arise. First problem with a tire model is that\textquotesingle s heavily dependent on more variables than initially taken into account in this research. A quick grasp would be environment variable like temperature, riding surface and humidity and although these might be determined by adding more sensors to the vehicle there are also variables which are more difficult to determine. Variables like how worn the tires are for instance. A way of determining the model at any given instant of time might be a more viable approach. This raises new problems Though. To determine data points throughout the entire spectrum the car needs to gather data at high slip ratio's at that given time. Something which can not always be done that easily. Drifting in a donut before driving off to work using your new safety systems every day doesn't seem that viable. Further research into generating a model without a lot of especially high slip ratio data points needs to be done to make this possible.