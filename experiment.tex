\section{Experimental}
 

In order to determine the tire characteristics of the scaled vehicle, a good testbed is necessary. In this section the experimental setup used to gather the data needed for determining the tire characteristic is discussed.\\
The experimental setup consists of a scaled RC car that was modified by previous BEP groups, and eventually by us. This vehicle is driven in a motion capture environment, namely the Mocap. This Mocap system is located in the DCSC (Delft Center for Systems and Control) Network Embedded Robotics Lab at Delft University of Technology.
It provides very accurate data (to the millimeter) with low latency. 

The scaled RC car is a Losi TEN Rally-X. It is a 1:10 scale car with 4WD. As told above, the car was modified by the previous groups. Here is a summation of the modifications done by the previous groups. 
First of all, the car was fitted with Hall sensors in the tires. These Hall sensors measure the revolutions per tire. Secondly, in order to fit these sensors, a new set of tires had to be 3D-printed because the standard tires didn\textquotesingle t offer enough space for the Hall sensors. Another important reason why the scaled rally car uses 3D-printed tires is because the standard tires had large deformations. Finally, the car\textquotesingle s suspension is replaced with 3D printed rods to eliminate the degrees of freedom of roll and pitch, to meet the assumptions of the Bicycle model. 

Unfortunately, there were several problems with the car and data acquisition. Therefore we did some modifications ourselves. Our modifications

The experiments that we conducted can be classified into three groups. The first set of experiments are the ones on the straight (longitudinal motion). During these we accelerated and braked while driving straight ahead. The second set are the steady state cornering experiments (lateral motion). Steady state cornering means cornering at a constant longitudinal velocity and constant steering angle. The first group of experiments focuses on longitudinal forces and slip ratios, while the second group focuses on lateral forces and slip angles. The tests were separated in order to distinguish longitudinal and lateral motion (Recommendation Barys Shyrokau). Finally, the third set of experiments are of combined motion. These tests take both lateral and longitudinal motion into account. 
	Variables of the tests are acceleration/deceleration for longitudinal motion, longitudinal velocity and steering angle for lateral motion, and these three combined for  combined motion. The variables were slightly increased each experiment in order to determine the \textquotesingle borderline\textquotesingle  between linear and nonlinear behaviour is. 

