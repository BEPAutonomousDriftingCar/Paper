\section{Introduction}
If you talk to racing enthusiasts you quickly find out one of the key components in any competition car are the tires. They provide the interface between all the components a race car consists of and the track it\textquotesingle s driving on. Therefore, tires dictate any behaviour a car is capable of. Those 4 foot sized patches of rubber make the difference between being able to make those quick sharp corners and spinning out of the ring. In daily traffic making those tires stick to the road can sometimes make the difference between life and death. In the United States alone, the National Highway Traffic Safety Administration (NHTSA) reports that 5 million crashes occurred in 2009 [59], causing over 30,000 deaths and 2 million injuries. 

Until now, driver safety systems in modern-cars have focussed on keeping the car in the linear regime while driving. For example, ABS prevents the wheels from slipping and traction control prevents the car from \textquotesingle breaking out\textquotesingle  of its path. However, when we look towards professional rally drivers we see that operating in the linear regime isn\textquotesingle t specifically necessary for driving safe. If we would be able to give the skills of a rally driver to driver assistance systems, or in the future autonomous cars, more accidents might be prevented. It would give forementioned systems more capabilities and options. Which might result in better accident prevention solutions.

This paper focuses on establishing a measurement setup and determining the tire characteristics of a small-scale RC car with it. It is a step towards achieving a good Nonlinear Model Predictive Control (NMPC) system. NMPC is a new control method which uses a dynamical model that can predict the motion of the car. Therefore, this system is able to let the car make evasive maneuvers up to the nonlinear regime of driving. In order to obtain a good dynamical model for the NMPC, tire characteristics are required. There are a few ways to obtain these characteristics. Normally, these characteristics are determined using special test rigs where only the wheels are placed in. However, these characteristics are only valid for that given tire, in only that given condition( think of temperature, wornness and wetness of the tire) on only a single given surface. A lot of testing can be done to determine the characteristics of all tires in all conditions. Yet it would be far better if a car would be able to determine it\textquotesingle s characteristics using on board sensors such as GPS and Inertial Measurment Units or IMU\textquotesingle s. Testing with real cars and on-board sensors in a special test environment is a possibility. But developing and refining this technique on small scale RC cars first requires less resources such as large test tracks and expensive cars and tires. Let alone other expenses like experienced mechanics and drivers.Therefore, using a small-scale RC car equipped with sensors is by far the least expensive way to obtain these tire characteristics. These methods can then be scaled for actual cars.

This all concludes to the aim of this paper, which is determining the tire characteristics of a small scale RC car using only on board sensors. At first, some scientific background is given on the subject. After that, the experimental setup and data processing will be explained. Next, results will be discussed. Finally, the conclusion will be given. The paper also leaves some room for recommendations and acknowledgements.  

