\section{Introduction}
Lorem ipsum dolor sit amet, consectetuer adipiscing elit. Maecenas porttitor congue massa. Fusce posuere, magna sed pulvinar ultricies, purus lectus malesuada libero, sit amet commodo magna eros quis urna.
Nunc viverra imperdiet enim. Fusce est. Vivamus a tellus.
Pellentesque habitant morbi tristique senectus et netus et malesuada fames ac turpis egestas. Proin pharetra nonummy pede. Maur
is et orci.
In de galerieën op het tabblad Invoegen bevinden zich items die zodanig zijn ontworpen dat deze bij het algemene uiterlijk van uw document passen. U kunt deze galerieën gebruiken om tabellen, kopteksten, voetteksten, lijsten, voorbladen en andere bouwstenen voor documenten in te voegen. Als u afbeeldingen, grafieken of diagrammen maakt, worden deze aangepast aan het huidige uiterlijk van uw document.
U kunt de opmaak van in de documenttekst geselecteerde tekst gemakkelijk wijzigen door een uiterlijk voor de geselecteerde tekst te kiezen in de galerie Snelle stijlen op het tabblad Start. U kunt de tekst ook rechtstreeks opmaken met de andere besturingselementen op het tabblad Start. Voor de meeste besturingselementen hebt u de keuze uit het uiterlijk van het huidige thema of een opmaak die u zelf opgeeft.
Als u het algemene uiterlijk van uw document wilt wijzigen, dient u nieuwe thema-elementen op het tabblad Pagina-indeling te kiezen. Gebruik de opdracht Huidige reeks snelle stijlen wijzigen om de beschikbare stijlen in de galerie Snelle stijlen te wijzigen. In zowel de galerie Thema's als de galerie Snelle stijlen bevinden zich opdrachten waarmee u het uiterlijk van het document altijd kunt terugzetten naar het oorspronkelijke uiterlijk van de huidige sjabloon.