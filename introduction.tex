\section{Introduction}
We live in a world where a lot of car accidents occur. This is something everybody knows. A less known fact is that many of these accidents could have been avoided. This is because of the fact that a lot of these accidents happen due to the inability of divers to control the car at the limits of friction. Race drivers are taken as a reference point. They can handle the car at the limits of the tires without losing control. So if autonomous vehicles or driver assistance systems have these capabilities, a lot of accidents can be avoided. 
	Until now, driver safety systems in modern-cars have focussed on keeping the car in the linear regime while driving. For example, ABS prevents the wheels from slipping and traction control prevents the car from ‘breaking out’ of its path. These systems keep the car in the linear regime. However, there is a serious need for safety systems which can operate the car in the nonlinear regime of driving, i.e., driving while the tires are slipping. Making evasive maneuvers is often the only way to prevent a collision, and this cannot be done without driving at the limits of friction.
